\documentclass[]{article}
\usepackage{lmodern}
\usepackage{amssymb,amsmath}
\usepackage{ifxetex,ifluatex}
\usepackage{fixltx2e} % provides \textsubscript
\ifnum 0\ifxetex 1\fi\ifluatex 1\fi=0 % if pdftex
  \usepackage[T1]{fontenc}
  \usepackage[utf8]{inputenc}
\else % if luatex or xelatex
  \ifxetex
    \usepackage{mathspec}
  \else
    \usepackage{fontspec}
  \fi
  \defaultfontfeatures{Ligatures=TeX,Scale=MatchLowercase}
\fi
% use upquote if available, for straight quotes in verbatim environments
\IfFileExists{upquote.sty}{\usepackage{upquote}}{}
% use microtype if available
\IfFileExists{microtype.sty}{%
\usepackage{microtype}
\UseMicrotypeSet[protrusion]{basicmath} % disable protrusion for tt fonts
}{}
\usepackage[margin=1in]{geometry}
\usepackage{hyperref}
\hypersetup{unicode=true,
            pdftitle={DaigleInClassLabWk14D1.R},
            pdfauthor={2011home},
            pdfborder={0 0 0},
            breaklinks=true}
\urlstyle{same}  % don't use monospace font for urls
\usepackage{color}
\usepackage{fancyvrb}
\newcommand{\VerbBar}{|}
\newcommand{\VERB}{\Verb[commandchars=\\\{\}]}
\DefineVerbatimEnvironment{Highlighting}{Verbatim}{commandchars=\\\{\}}
% Add ',fontsize=\small' for more characters per line
\usepackage{framed}
\definecolor{shadecolor}{RGB}{248,248,248}
\newenvironment{Shaded}{\begin{snugshade}}{\end{snugshade}}
\newcommand{\KeywordTok}[1]{\textcolor[rgb]{0.13,0.29,0.53}{\textbf{#1}}}
\newcommand{\DataTypeTok}[1]{\textcolor[rgb]{0.13,0.29,0.53}{#1}}
\newcommand{\DecValTok}[1]{\textcolor[rgb]{0.00,0.00,0.81}{#1}}
\newcommand{\BaseNTok}[1]{\textcolor[rgb]{0.00,0.00,0.81}{#1}}
\newcommand{\FloatTok}[1]{\textcolor[rgb]{0.00,0.00,0.81}{#1}}
\newcommand{\ConstantTok}[1]{\textcolor[rgb]{0.00,0.00,0.00}{#1}}
\newcommand{\CharTok}[1]{\textcolor[rgb]{0.31,0.60,0.02}{#1}}
\newcommand{\SpecialCharTok}[1]{\textcolor[rgb]{0.00,0.00,0.00}{#1}}
\newcommand{\StringTok}[1]{\textcolor[rgb]{0.31,0.60,0.02}{#1}}
\newcommand{\VerbatimStringTok}[1]{\textcolor[rgb]{0.31,0.60,0.02}{#1}}
\newcommand{\SpecialStringTok}[1]{\textcolor[rgb]{0.31,0.60,0.02}{#1}}
\newcommand{\ImportTok}[1]{#1}
\newcommand{\CommentTok}[1]{\textcolor[rgb]{0.56,0.35,0.01}{\textit{#1}}}
\newcommand{\DocumentationTok}[1]{\textcolor[rgb]{0.56,0.35,0.01}{\textbf{\textit{#1}}}}
\newcommand{\AnnotationTok}[1]{\textcolor[rgb]{0.56,0.35,0.01}{\textbf{\textit{#1}}}}
\newcommand{\CommentVarTok}[1]{\textcolor[rgb]{0.56,0.35,0.01}{\textbf{\textit{#1}}}}
\newcommand{\OtherTok}[1]{\textcolor[rgb]{0.56,0.35,0.01}{#1}}
\newcommand{\FunctionTok}[1]{\textcolor[rgb]{0.00,0.00,0.00}{#1}}
\newcommand{\VariableTok}[1]{\textcolor[rgb]{0.00,0.00,0.00}{#1}}
\newcommand{\ControlFlowTok}[1]{\textcolor[rgb]{0.13,0.29,0.53}{\textbf{#1}}}
\newcommand{\OperatorTok}[1]{\textcolor[rgb]{0.81,0.36,0.00}{\textbf{#1}}}
\newcommand{\BuiltInTok}[1]{#1}
\newcommand{\ExtensionTok}[1]{#1}
\newcommand{\PreprocessorTok}[1]{\textcolor[rgb]{0.56,0.35,0.01}{\textit{#1}}}
\newcommand{\AttributeTok}[1]{\textcolor[rgb]{0.77,0.63,0.00}{#1}}
\newcommand{\RegionMarkerTok}[1]{#1}
\newcommand{\InformationTok}[1]{\textcolor[rgb]{0.56,0.35,0.01}{\textbf{\textit{#1}}}}
\newcommand{\WarningTok}[1]{\textcolor[rgb]{0.56,0.35,0.01}{\textbf{\textit{#1}}}}
\newcommand{\AlertTok}[1]{\textcolor[rgb]{0.94,0.16,0.16}{#1}}
\newcommand{\ErrorTok}[1]{\textcolor[rgb]{0.64,0.00,0.00}{\textbf{#1}}}
\newcommand{\NormalTok}[1]{#1}
\usepackage{graphicx,grffile}
\makeatletter
\def\maxwidth{\ifdim\Gin@nat@width>\linewidth\linewidth\else\Gin@nat@width\fi}
\def\maxheight{\ifdim\Gin@nat@height>\textheight\textheight\else\Gin@nat@height\fi}
\makeatother
% Scale images if necessary, so that they will not overflow the page
% margins by default, and it is still possible to overwrite the defaults
% using explicit options in \includegraphics[width, height, ...]{}
\setkeys{Gin}{width=\maxwidth,height=\maxheight,keepaspectratio}
\IfFileExists{parskip.sty}{%
\usepackage{parskip}
}{% else
\setlength{\parindent}{0pt}
\setlength{\parskip}{6pt plus 2pt minus 1pt}
}
\setlength{\emergencystretch}{3em}  % prevent overfull lines
\providecommand{\tightlist}{%
  \setlength{\itemsep}{0pt}\setlength{\parskip}{0pt}}
\setcounter{secnumdepth}{0}
% Redefines (sub)paragraphs to behave more like sections
\ifx\paragraph\undefined\else
\let\oldparagraph\paragraph
\renewcommand{\paragraph}[1]{\oldparagraph{#1}\mbox{}}
\fi
\ifx\subparagraph\undefined\else
\let\oldsubparagraph\subparagraph
\renewcommand{\subparagraph}[1]{\oldsubparagraph{#1}\mbox{}}
\fi

%%% Use protect on footnotes to avoid problems with footnotes in titles
\let\rmarkdownfootnote\footnote%
\def\footnote{\protect\rmarkdownfootnote}

%%% Change title format to be more compact
\usepackage{titling}

% Create subtitle command for use in maketitle
\newcommand{\subtitle}[1]{
  \posttitle{
    \begin{center}\large#1\end{center}
    }
}

\setlength{\droptitle}{-2em}
  \title{DaigleInClassLabWk14D1.R}
  \pretitle{\vspace{\droptitle}\centering\huge}
  \posttitle{\par}
  \author{2011home}
  \preauthor{\centering\large\emph}
  \postauthor{\par}
  \predate{\centering\large\emph}
  \postdate{\par}
  \date{Tue Apr 24 15:09:12 2018}


\begin{document}
\maketitle

\begin{Shaded}
\begin{Highlighting}[]
\CommentTok{# Chris Daigle}
\CommentTok{# Week 14 D1 In Class Lab}

\CommentTok{# 1. Take a derivative of the cdf of N(2, 2^2) at x = 0}
\CommentTok{# pnorm(x, mean, sd, ...) is the cdf of normal}
\NormalTok{f <-}\StringTok{ }\ControlFlowTok{function}\NormalTok{(x) \{}
\NormalTok{  f <-}\StringTok{ }\KeywordTok{pnorm}\NormalTok{(x, }\DataTypeTok{mean =} \DecValTok{2}\NormalTok{, }\DataTypeTok{sd =} \DecValTok{4}\NormalTok{)}
  \KeywordTok{return}\NormalTok{(f)}
\NormalTok{\}}

\NormalTok{f_dev <-}\StringTok{ }\ControlFlowTok{function}\NormalTok{(x, f) \{}
\NormalTok{  h <-}\StringTok{ }\FloatTok{1e-8}
\NormalTok{  f_dev <-}\StringTok{ }\NormalTok{(}\KeywordTok{f}\NormalTok{(x }\OperatorTok{+}\StringTok{ }\NormalTok{h) }\OperatorTok{-}\StringTok{ }\KeywordTok{f}\NormalTok{(x)) }\OperatorTok{/}\StringTok{ }\NormalTok{h}
  \KeywordTok{return}\NormalTok{(f_dev)}
\NormalTok{\}}

\KeywordTok{f_dev}\NormalTok{(}\DataTypeTok{x =} \DecValTok{0}\NormalTok{, }\DataTypeTok{f =}\NormalTok{ f)}
\end{Highlighting}
\end{Shaded}

\begin{verbatim}
## [1] 0.08801633
\end{verbatim}

\begin{Shaded}
\begin{Highlighting}[]
\CommentTok{# 2. Calculate the volume of half sphere with a radius of 1.}
\CommentTok{# General form: int_\{a\}^\{b\} pi*sqrt(r^2-x^2)dx = 4*pi((r^3)/3)}

\NormalTok{x <-}\StringTok{ }\KeywordTok{c}\NormalTok{(}\DecValTok{0}\NormalTok{,}\DecValTok{1}\NormalTok{) }\CommentTok{# for half and not the whole, if whole, c(-1,1)}
\NormalTok{h <-}\StringTok{ }\FloatTok{0.0000001}
\NormalTok{r <-}\StringTok{ }\DecValTok{1}

\NormalTok{s <-}\StringTok{ }\ControlFlowTok{function}\NormalTok{(r,x) \{}
  \KeywordTok{return}\NormalTok{(pi}\OperatorTok{*}\NormalTok{((r }\OperatorTok{^}\StringTok{ }\DecValTok{2}\NormalTok{) }\OperatorTok{-}\StringTok{ }\NormalTok{(x }\OperatorTok{^}\StringTok{ }\DecValTok{2}\NormalTok{)))}
\NormalTok{\}}

\NormalTok{sphere <-}\StringTok{ }\ControlFlowTok{function}\NormalTok{(x, s, h) \{}
\NormalTok{  volume <-}\StringTok{ }\DecValTok{0}
  \ControlFlowTok{for}\NormalTok{ (i }\ControlFlowTok{in} \KeywordTok{seq}\NormalTok{(}\DataTypeTok{from =}\NormalTok{ x[}\DecValTok{1}\NormalTok{], }\DataTypeTok{to =}\NormalTok{ x[}\DecValTok{2}\NormalTok{] }\OperatorTok{-}\StringTok{ }\NormalTok{h, }\DataTypeTok{by =}\NormalTok{ h)) \{}
\NormalTok{    volume <-}\StringTok{ }\NormalTok{volume }\OperatorTok{+}\StringTok{ }\NormalTok{h}\OperatorTok{*}\KeywordTok{s}\NormalTok{(r,i)}
\NormalTok{  \}}
  \KeywordTok{return}\NormalTok{(volume)}
\NormalTok{\}}
\KeywordTok{sphere}\NormalTok{(x, s, h)}
\end{Highlighting}
\end{Shaded}

\begin{verbatim}
## [1] 2.094395
\end{verbatim}

\begin{Shaded}
\begin{Highlighting}[]
\CommentTok{# Check}
\NormalTok{(}\DecValTok{1}\OperatorTok{/}\DecValTok{2}\NormalTok{)}\OperatorTok{*}\NormalTok{(}\DecValTok{4}\OperatorTok{*}\NormalTok{pi}\OperatorTok{*}\NormalTok{((r}\OperatorTok{^}\DecValTok{2}\NormalTok{)}\OperatorTok{/}\DecValTok{3}\NormalTok{))}
\end{Highlighting}
\end{Shaded}

\begin{verbatim}
## [1] 2.094395
\end{verbatim}

\begin{Shaded}
\begin{Highlighting}[]
\CommentTok{# 3. Find maximizer of the following}
\CommentTok{# Suppose that we allocate our budget between online and TV advertisements to maximize revenue}
\CommentTok{# (1) The effects on revenue for each advertisement is $200 and $600}
\CommentTok{# (2) The cost 1 for each advertisement is $150 and $100 and total expense for this cannot be more than $10000.}
\CommentTok{# (3) The cost 2 for each advertisement is $50 and $300 and total expense for this cannot be more than $10000.}
\CommentTok{# (4) At least we should have 95 advertisements in total.}

\CommentTok{# I interpret this as two scenarios: cost 1 and cost 2 scenario}

\KeywordTok{library}\NormalTok{(}\StringTok{'lpSolve'}\NormalTok{)}
\CommentTok{#Scenario 1:}
\CommentTok{# maxπ = 200X+600Y - 150X - 100Y = 50X+500Y}
\CommentTok{# s.t.}
\CommentTok{# 150X + 100Y <= 10,000}
\CommentTok{# X + Y >= 95}

\NormalTok{obj.fun <-}\StringTok{ }\KeywordTok{c}\NormalTok{(}\DecValTok{50}\NormalTok{, }\DecValTok{500}\NormalTok{)}
\NormalTok{const <-}\StringTok{ }\KeywordTok{matrix}\NormalTok{(}\KeywordTok{c}\NormalTok{(}\DecValTok{150}\NormalTok{, }\DecValTok{100}\NormalTok{, }\DecValTok{1}\NormalTok{, }\DecValTok{1}\NormalTok{), }\DataTypeTok{ncol =} \DecValTok{2}\NormalTok{, }\DataTypeTok{byrow =} \OtherTok{TRUE}\NormalTok{)}
\NormalTok{const.dir <-}\StringTok{ }\KeywordTok{c}\NormalTok{(}\StringTok{'<='}\NormalTok{, }\StringTok{'>='}\NormalTok{)}
\NormalTok{rhs <-}\StringTok{ }\KeywordTok{c}\NormalTok{(}\DecValTok{10000}\NormalTok{, }\DecValTok{95}\NormalTok{)}

\NormalTok{Scenario1 <-}\StringTok{ }\KeywordTok{lp}\NormalTok{(}\StringTok{'max'}\NormalTok{, obj.fun, const, const.dir, rhs)}
\NormalTok{Scenario1}\OperatorTok{$}\NormalTok{solution}
\end{Highlighting}
\end{Shaded}

\begin{verbatim}
## [1]   0 100
\end{verbatim}

\begin{Shaded}
\begin{Highlighting}[]
\CommentTok{#}
\CommentTok{# Scenario 2:}
\CommentTok{# maxπ = 200X+600Y - 50X - 300Y = 150X+300Y}
\CommentTok{# s.t.}
\CommentTok{# 50X + 300Y <= 10,000}
\CommentTok{# X + Y >= 95}
\NormalTok{obj.fun <-}\StringTok{ }\KeywordTok{c}\NormalTok{(}\DecValTok{150}\NormalTok{, }\DecValTok{300}\NormalTok{)}
\NormalTok{const <-}\StringTok{ }\KeywordTok{matrix}\NormalTok{(}\KeywordTok{c}\NormalTok{(}\DecValTok{50}\NormalTok{, }\DecValTok{300}\NormalTok{, }\DecValTok{1}\NormalTok{, }\DecValTok{1}\NormalTok{), }\DataTypeTok{ncol =} \DecValTok{2}\NormalTok{, }\DataTypeTok{byrow =} \OtherTok{TRUE}\NormalTok{)}
\NormalTok{const.dir <-}\StringTok{ }\KeywordTok{c}\NormalTok{(}\StringTok{'<='}\NormalTok{, }\StringTok{'>='}\NormalTok{)}
\NormalTok{rhs <-}\StringTok{ }\KeywordTok{c}\NormalTok{(}\DecValTok{10000}\NormalTok{, }\DecValTok{95}\NormalTok{)}

\NormalTok{Scenario2 <-}\StringTok{ }\KeywordTok{lp}\NormalTok{(}\StringTok{'max'}\NormalTok{, obj.fun, const, const.dir, rhs)}
\NormalTok{Scenario2}\OperatorTok{$}\NormalTok{solution}
\end{Highlighting}
\end{Shaded}

\begin{verbatim}
## [1] 200   0
\end{verbatim}


\end{document}
